\documentclass[a4paper,12pt,
		      12pt,
		      DIV=12,
		      parskip=half,
		      oneside]{scrbook}

\frenchspacing
\pagestyle{myheadings}

\usepackage[utf8]{inputenc}
\usepackage[T1]{fontenc}
\usepackage{lmodern}
\usepackage[intlimits]{amsmath} 
\usepackage{amsfonts}
\usepackage{amssymb}
\usepackage{esint}
\usepackage[dvips]{graphicx}
\usepackage{epsfig}
\usepackage{psfrag}
\usepackage[numbers,square]{natbib}
\bibliographystyle{unsrtdin}
\usepackage[pdfborder={0 0 0}]{hyperref} % Try to comment this out, if compiling crashes.
\usepackage{breakurl}
\usepackage{color} 
\definecolor{gray}{rgb}{.6,.6,.6}
\definecolor{darkgray}{rgb}{.4,.4,.4}
\definecolor{brightgray}{rgb}{.9,.9,.9}
\definecolor{darkblue}{rgb}{0,0,.8}
\definecolor{darkgreen}{rgb}{0,.7,0}
\definecolor{brightyellow}{rgb}{1,1,.5}

\usepackage{tabularx}
\usepackage{ragged2e}
\newcolumntype{Y}{>{\RaggedRight\arraybackslash}X}
\usepackage{colortbl}

\frenchspacing
\newcommand{\clearcustom}{\cleardoublepage} 

\usepackage{listings} 
\lstset{
  language=matlab,             	% the language of the code
  basicstyle=\ttfamily,		% the size of the fonts that are used for the code \footnotesize,          
  numbersep=5pt, 		% where to put the line-numbers % how far the line-numbers are from the code
  numbers=left,
  numberstyle=\tiny\color{gray},  % the style that is used for the line-numbers,
  numberbychapter=true,
  frame=lRtb,			% adds a frame around the code.  single
  rulecolor=\color{darkblue}, 	% if not set, the frame-color may be changed on line-breaks within not-black text (e.g. commens (green here))
  tabsize=2,			% sets default tabsize to 2 spaces
  showstringspaces=false,	% underline spaces within strings      
%    stringstyle=\color{red},
    commentstyle=\color{gray},
%   identifierstyle=\color{darkblue},
%   keywordstyle=\color{darkgreen},
  captionpos=b,          	% sets the caption-position to bottom
  stepnumber=1,                   % the step between two line-numbers. If it's 1, each line 
                                  % will be numbered
  backgroundcolor=\color{white},      % choose the background color. You must add \usepackage{color}
  showspaces=false,               % show spaces adding particular underscores                                  
  breaklines=true,                % sets automatic line breaking
  breakatwhitespace=true,        % sets if automatic breaks should only happen at whitespace
%   title=\lstname,                 % show the filename of files included with \lstinputlisting;
                                  % also try caption instead of title
%   escapeinside={\%*}{*)},            % if you want to add a comment within your code
%   morekeywords={*,...}               % if you want to add more keywords to the set
} 
%%%%%%%%%%%%%%%%%%%%%%%%%%%%for listing shell in linux 
\lstdefinestyle{shell}{
language=sh,
delim=[il][\bfseries]{BB}
} % for code in linux. for examp. :\begin{lstlisting}[style=shell] ....\end{listings}



\DeclareMathOperator*{\rot}{rot}
\DeclareMathOperator*{\divergence}{div}
\DeclareMathOperator*{\grad}{grad}
\newcommand{\corresponds}{\stackrel{\scriptscriptstyle\wedge}{=}}
\newcommand{\defineequal}{\stackrel{!}{=}}
\newcommand{\defineequivalent}{\stackrel{!}{\Leftrightarrow}}

\newcommand{\iconbox}[2]{%
\fbox{
\begin{minipage}{0.1\textwidth}
 \includegraphics[width=\textwidth]{#1}
\end{minipage}
\begin{minipage}{0.9\textwidth}
#2
\end{minipage}}}

\newcommand{\warning}[1]{\iconbox{svg/Warning.eps}{#1}}
\newcommand{\info}[1]{\iconbox{svg/Info.eps}{#1}}
\newcommand{\incomplete}{\warning{\textbf{Warning: This chapter or section is incomplete.}\\
Visit \url{http://openEMS.de} how to contribute and enhance this User Manual.}}

\usepackage{bm}
\usepackage{float}
\usepackage{subfig}

\usepackage{environ}% http://ctan.org/pkg/environ

\newenvironment{myindentpar}[1][1em]% INDENT. 
     { \begin{list}{}{\setlength{\leftmargin}{#1}}%
              \item[]% just blank 
     }
     {\end{list}}

%%% for the font of the title of a function
\definecolor{funcnamecol}{RGB}{200,0,0}
\newenvironment{FontNameFunct}[1] % \begin{FontNameFunct}{function-name}  \end{FontNameFunct}
{\begin{Large}\textcolor{funcnamecol}{\texttt{\textbf{{#1}}}}}{\end{Large}}

%%% the environment of  purpose,syntax, definition, description, example.... #1 is the name of title.
\newenvironment{FontDescr}[1]% \begin{FontDescr}{title} context \end{FontDescr}
  {    \begin{myindentpar}%{1em}
	\begin{normalsize}\textcolor{black}{\textbf{{#1}}}\end{normalsize}
	\end{myindentpar}
	\begin{myindentpar}[2em]
  }
  {
	\end{myindentpar}
  }

%%% font of  matlab variables including links to the first definition.
\definecolor{varcol}{RGB}{0,127,255}
\usepackage{ifthen}
\newcommand{\matv}[2][\empty]    %% \matv[label]{parameter-name}.   if the name of the variable has underscore '_' , for example a_b, \matv[para:a_b]{a\_b}
  {
      \ifthenelse{\equal{#1}{\empty}}  % first optional argument is the label that the variable refer to.  #2 is for the name of the variable. default #1 is para:#2. \matv{paramete-rname}=\matv[para:parameter-name]{parameter-name}
      {\hyperref[para:#2]{\textcolor{varcol}{\texttt{#2}}}}
      {\hyperref[#1]{\textcolor{varcol}{\texttt{#2}}}}
}

%%% font of the possible values that can be assigned to the arguments of a function in matlab values
\newcommand{\matval}[1]{\textcolor{green}{\texttt{#1}}} % \matval{0} for Cartesian coordinates system.

%%%  the environment of the title of parameter
\newenvironment{FontPara}[2][\empty] % \begin{FontPara}[label]{title} context \end{FontPara}  . Default label is para:#2
  {  \par  \matv[#1]{#2}
    \begin{myindentpar}%[3em]
  }
  {  \end{myindentpar}
  }

