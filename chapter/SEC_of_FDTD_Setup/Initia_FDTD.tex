%%%%%%%%%%%%%%%%%%%%%%%%%%%%%section Attributes of FDTD %%%%%%%%%%%%%%%%%%%%%%%%%%%%%%%%%%%%%%%%%%%%%%%%%%%%%%%%%%%%
%%%%%%%%%%%%%%%%%%%%%%%%%%%%%%%%%%%%%%%%%%%%%%%%%%%%%%%%%%%%%%%%%%%%%%%%%%%%%%%%%%%%%%%%%%%%%%%%%%%%%%%%%%%%%%%%%%
\section{Initiation of FDTD} \label{sec:FDTD_ATTRIBUTE}
\phantomsection \label{para:FDTD_ATTRIBUTE}
This section shows how to initiate the structure \matv{FDTD} with some given attributes.
 
%%%%%%%%%%%%%%%%%%%%%%% FUNCTION  InitFDTD %%%%%%%%%%%%%%%%%%%%%%%%%%%%%%%%%%%5
\begin{FontNameFunct}{InitFDTD()}
\phantomsection \label{func:InitFDTD} 
\end{FontNameFunct}

%%%%%%%%%%%%%%%%%%%%%%%%%%%% Purpose
\begin{FontDescr}{Purpose:}
Initialize the \matv{FDTD} data-structure.
\end{FontDescr}

%%%%%%%%%%%%%%%%%%%%%%%%%%%% Syntax
\begin{FontDescr}{Syntax:}
      \begin{lstlisting}
FDTD = InitFDTD(NrTS,endCrit,varargin)
      \end{lstlisting}
\end{FontDescr}

% \begin{FontNameDescr}{Definition:}
% Return value of this function is a initial structure of \matv{FDTD}. Many parameters, such as \matv{NrTS} the max. number of timesteps to simulate,
% \end{FontNameDescr}
%%%%%%%%%%%%%%%%%%%%%%%%%%%% Description
\begin{FontDescr}{Description:}
Return value of this function is a initial structure  \matv{FDTD}. A group of parameters about the algorithm FDTD can be assigned by this function. This group of parameters are about end  simulation criteria respectively  using time, time-steps (default=1e9) or energy (default=1e-5). Furthermore, Nyquist oversampling of time domain dump, coordinate system and multi-grids \cite{EC_FDTD_CYL_THORSTEN} can be also set by this function.
    %%%%%%%%%%%%%%%%%%%%%%%%%%%% NrTS
    \begin{FontPara}{NrTS} \phantomsection \label{para:NrTS}
    Maximal number of time-steps to simulate (e.g. default=1e9). When the simulation runs to this maximal time-step, the simulation will stop.
    \end{FontPara}
    %%%%%%%%%%%%%%%%%%%%%%%%%%%% endCrit
    \begin{FontPara}{endCrit} \phantomsection \label{para:endCrit}
    End criteria, e.g. 1e-5. Simulations stops if energy has decayed by this value (<1e-4 is recommended, default=1e-5). Sometimes it can be set greater, for example 1e-2, only to check whether the code runs correctly.
    \end{FontPara}
\end{FontDescr}

\begin{FontDescr}{Optional Arguments:}
%     %%%%%%%%%%%%%%%%%%%%%%%%%%%% FDTD
%     \begin{FontPara}{FDTD}
%     The initialized structure for the parameters of the algorithm FDTD. See \matv{FDTD}.
%     \end{FontPara}
    %%%%%%%%%%%%%%%%%%%%%%%%%%%% MaxTime
    \begin{FontPara}{MaxTime} \phantomsection \label{para:MaxTime}
    Maximal real time in seconds to simulate.
    \end{FontPara}
    %%%%%%%%%%%%%%%%%%%%%%%%%%%% OverSampling
    \begin{FontPara}{OverSampling} \phantomsection \label{para:OverSampling}
    For Nyquist oversampling of time domain dump. 10 \textasciitilde 20 is recommended. See \cite{Nyquist-Shannon_sampling}. 
    \end{FontPara}
    %%%%%%%%%%%%%%%%%%%%%%%%%%%% CoordSystem
    \begin{FontPara}{CoordSystem} \phantomsection \label{para:CoordSystem}
    Choose coordinate system (\matval{0} for Cartesian coordinate system, \matval{1} for cylindrical coordinate system).
    \end{FontPara}
    %%%%%%%%%%%%%%%%%%%%%%%%%%%% endCrit
    \begin{FontPara}{MultiGrid} \phantomsection \label{para:MultiGrid}
    \matv{MultiGrid} is a string. It can set   the grids  more homogeneous in direction $\alpha$ for different $\rho$ in cylindrical coordinate system. 
    \end{FontPara}
\end{FontDescr}

%%%%%%%%%%%%%%%%%%%%% Examples.
	\begin{FontDescr}{Examples:}
\begin{itemize}
\item 
\begin{lstlisting}
% Default initiation with 1e9 as max. timesteps and -50dB as end-criteria
FDTD = InitFDTD(); 
\end{lstlisting}

\item 
\begin{lstlisting}
% Initiation with 1e6 as max. timesteps and -60dB as end-criteria
FDTD = InitFDTD(1e6, 1e-6);
\end{lstlisting}

\item 
\begin{lstlisting}
% Using oversampling and cylindrical coordinate system for FDTD simulation.
FDTD = InitFDTD(1e6, 1e-6,...
               'CoordSystem', 1,'OverSampling',10);
\end{lstlisting}

\item 
\begin{lstlisting}
%% An example using multigrids.
% radius where the number of lines in alpha direction begin to be doupel.
MultiGrid_Level = [100 400];
% create comma-separated string
mg_str = num2str(MultiGrid_Level,'%d,');
% If the number of lines in alpha direction for rho <= 100 is n_y, then the number for 100 <rho < =400 is 2*n_y and the number for rho > 400  is 2^2*n_y.
FDTD =InitFDTD('MultiGrid',mg_str(1:end-1),...
               'CoordSystem',1);
\end{lstlisting}
\end{itemize}	 
	\end{FontDescr}


    %%%%%%%%%%%%%% warning
	      \warning{ Please don't neglect the left expression '\matv{FDTD}' for the assignment, otherwise \matv{FDTD} will not be updated.}

% See also: \matv[func:InitCSX]{InitCSX}
