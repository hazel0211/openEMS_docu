%Dispersive Material
User can model a Drude type dispersive material by function \texttt{AddLorentzMaterial()}. Dispersive material has frequency dependent relative permittivity($\varepsilon_{r}$) and permeability($\mu_{r}$), and they can be defined with function \texttt{ SetMaterialProperty()}.
\begin{equation}
\varepsilon_{r}(\omega)=\varepsilon_{r}*(1-\dfrac{\omega_{\varepsilon}^2}{\omega*(\omega-j/t_{\varepsilon})})
\end{equation}
\begin{equation}
\mu_{r}(\omega)=\mu_{r}*(1-\dfrac{\omega_{\mu}^2}{\omega*(\omega-j/t_{\mu})})
\end{equation}
where \begin{myindentpar}
\begin{itemize}
\item $\omega_{\varepsilon}$: the respective electric angular plasma frequency
\item $t_{\varepsilon}$: the electric relaxation time 
\item $\omega_{\mu}$:the respective magnetic angular plasma frequency
\item $t_{\mu}$:the magnetic relaxation time
\end{itemize}
\end{myindentpar}

\begin{FontNameFunct}{AddLorentzMaterial}
\end{FontNameFunct}

\begin{FontDescr}{Syntax:}
  \begin{lstlisting}
 CSX = AddLorentzMaterial(CSX, name)
  \end{lstlisting}
\end{FontDescr}

\begin{FontDescr}{Description:}
  This syntax adds a Drude type dispersive material model with the given name.\\ 
\end{FontDescr} 
  

\begin{FontNameFunct}{SetMaterialProperty}  
\end{FontNameFunct}
In this section the frequency dependent properties $\varepsilon_{r}(\omega)$ and $\mu_{r}(\omega)$ will be defined. Refer to \ref{subsection_material_prop} for non-dispersive material properties setup. 

 \begin{FontDescr}{Syntax:}
  \begin{lstlisting}
 CSX = SetMaterialProperty(CSX, name, varargin)
  \end{lstlisting}
 \end{FontDescr}
 
 \begin{FontDescr}{Arguments:}  
 
  \begin{FontPara}{EpsilonPlasmaFrequency}
  Electric plasma frequency($f_{\varepsilon}$). It equals to   $\omega_{\varepsilon}/2\pi$. 
  \end{FontPara} 
  \begin{FontPara}{MuePlasmaFrequency}($f_{\mu}$)
  Magnetic plasma frequency.It equals to $\omega_{\mu}/2\pi$.   
  \end{FontPara}
  \begin{FontPara}{EpsilonRelaxTime}   
   Electric plasma relaxation time($t_{\varepsilon}$). Smaller number results in greater losses or alternatively set it to '0' for lossless case. 
   \end{FontPara}
  \begin{FontPara}{MueRelaxTime}  
  Magnetic plasma relaxation time($t_{\mu}$).Smaller number results in greater losses or alternatively set it to '0' for lossless case.
 \end{FontPara}

For higher order Drude type, the above mentioned material constant is modified to: 

 \textcolor{varcol}{\texttt{EpsilonPlasmaFrequency$\_<n>$}}
\begin{myindentpar}
 n-th electric plasma frequency.
\end{myindentpar} 
  \textcolor{varcol}{\texttt{MuePlasmaFrequency$\_<n>$} }  
  \begin{myindentpar}
  n-th magnetic plasma frequency. 
  \end{myindentpar} 
  \textcolor{varcol}{\texttt{EpsilonRelaxTime$\_<n>$}} 
   \begin{myindentpar}
   n-th electric plasma relaxation time($t_{\varepsilon}$).   \end{myindentpar} 
 \textcolor{varcol}{\texttt{MueRelaxTime$\_<n>$}} 
\begin{myindentpar}
n-th magnetic plasma relaxation time($t_{\mu}$).
 \end{myindentpar}  
\end{FontDescr}  


\begin{FontDescr}{Example:}  
\begin{lstlisting} 
 CSX = AddLorentzMaterial(CSX,'drude');
 CSX = SetMaterialProperty(CSX,'drude','Epsilon',5,
 'EpsilonPlasmaFrequency',5e9,'EpsilonRelaxTime',1e-9);
 CSX = SetMaterialProperty(CSX,'drude','Mue',5,
 'MuePlasmaFrequency',5e9,'MueRelaxTime',1e-9);
\end{lstlisting}
  
  A Lorent material of name 'drude' has been defined. Its $\varepsilon_{r}$ and $\mu_{r}$ are set to 5 ,$f_{\varepsilon}$ and $f_{\mu}$ are both at 500M Hz while $t_{\varepsilon}$ and $t_{\mu}$ value 1e-9. 
 \end{FontDescr}
  
  
  