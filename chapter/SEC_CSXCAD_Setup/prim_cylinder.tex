Introduce a cylinder into\hyperref[CSX]{\matv{CSX}} with predefined matlab function \texttt{AddCylinder}.

\begin{FontNameFunct}{AddCylinder()}
\end{FontNameFunct}

\begin{FontDescr}{Purpose:}
Define cylinder with its axis(where it extends) and radius and assign a property to it. 
\end{FontDescr}

\begin{FontDescr}{Syntax:}
\begin{lstlisting} 
 CSX = AddCylinder(CSX, propName, prio, start, stop, rad, varargin)
\end{lstlisting}
\end{FontDescr}

\begin{FontDescr}{Description:}

\begin{FontPara}{propName}
Refer to \hyperref[prim_Name]{propName} in \texttt{AddBox}. 
\end{FontPara}

\begin{FontPara}{start}
A vector represents start point of cylinder axis.  
\end{FontPara}

\begin{FontPara}{stop}
End point of cylinder axis(vector). Extend in the same direction as start point.  
\end{FontPara}

\begin{FontPara}{rad}
Radius of cylinder. In drawing unit. 
\end{FontPara}
\end{FontDescr}

\begin{FontDescr}{Optional Arguments:}
The standard trasformation (rotation,translation,scaling) mentioned in  \hyperref[prim_transform]{\matv{'Transform'}} of \texttt{AddBox}.  
\end{FontDescr}

\begin{FontDescr}{Examples:}

\begin{lstlisting} 
CSX=AddMaterial(CSX,'phantom');
CSX=SetMaterialProperty(CSX,'phantom','Epsilon',75.5,
'Kappa',0.438,'Density',1000);
start=[0 0 -30 ];
stop=[0 0 30 ];
CSX=AddCylinder(CSX,'phantom',5,start,stop,10);
\end{lstlisting}
This example introduces a 10(drawing unit) radius homogeneous phantom into\hyperref[CSX]{\matv{CSX}}. This phantom has properties of $\varepsilon_{r}$=75.5, $\kappa$=0.438 and density of 1000. It extends in z-direction and has 60(drawing unit)height.  

\end{FontDescr}
