Introduce a cylindrical shell into\hyperref[CSX]{\matv{CSX}}. 

\begin{FontNameFunct}{AddCylindricalShell()}
\end{FontNameFunct}


\begin{FontDescr}{Purpose:}
To add cylinder with specified shell width to\hyperref[CSX]{\matv{CSX}}. 
\end{FontDescr}

\begin{FontDescr}{Syntax:}
\begin{lstlisting} 
 CSX = AddCylindricalShell(CSX, propName, prio, start, stop, rad, shell_width, varargin)
\end{lstlisting}
\end{FontDescr}

\begin{FontDescr}{Description:}
The parameters are defined same as those in \texttt{Addcylinder}(\ref{cylinder}) except the following: \\
\textcolor{varcol}{shell$\_$width}
\begin{myindentpar} Width of the shell. The inner radius($r_{in}$) and outer radius($r_{out}$) of shell are:

\begin{equation}    
r_{in}=rad-shell\_width/2 
\end{equation}
\label{rin}
\begin{equation}
r_{out}=rad+shell\_width/2 
\end{equation}
\label{rout}

\end{myindentpar} 

\end{FontDescr}

\begin{FontDescr}{Optional Arguments:}
The standard trasformation (rotation,translation,scaling) mentioned in  \hyperref[prim_transform]{\matv{'Transform'}} of \texttt{AddBox}.   
\end{FontDescr}

\begin{FontDescr}{Examples:}

\begin{lstlisting} 
CSX=AddMaterial(CSX,'plexi_shield');
CSX=SetMaterialProperty(CSX,'plexi_shield','Epsilon'
,2.22);
start=[0 0 -30 ];
stop=[0 0 30 ];
CSX=AddCylindricalShell(CSX,'plexi_shield',5,start,stop,
20,10);
\end{lstlisting}
A cylinder shell of radius 20 and shell width of 10 has defined around z-axis.It has dielectric material property($\varepsilon_{r}$=2.22) and height of 60. The inner radius of cylindrical shell is 20-10/2=15 ; the outer radius of it is 20+10/2=25 drawing unit.    
\end{FontDescr}

