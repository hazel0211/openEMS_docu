This section introduces 'Dumpbox' property to store $\mathbf{E}$,$\mathbf{H}$-field, electrical current , current density and SAR(Specific Absorption Rate) in a specified box.  


\begin{FontNameFunct}{AddDump()}
\end{FontNameFunct}

\begin{FontDescr}{Purpose:}
Add Dumpbox property into 
\matv{CSX}\phantomsection\label{CSX} with the given name. 
\end{FontDescr}

\begin{FontDescr}{Syntax:}
\begin{lstlisting} 
 CSX = AddDump(CSX, name, varargin)
\end{lstlisting}
\end{FontDescr}

\begin{FontDescr}{Description:}

\begin{FontPara}{DumpType}
\textcolor{green}{0}: time domain $\mathbf{E}$-field \\
\textcolor{green}{1}: time domain $\mathbf{H}$-field \\
\textcolor{green}{2}: time domain electrical current($\mathbf{I}$)  \\
\textcolor{green}{3}: time domain electrical current density($\vec{\mathbf{J}}$)\\

\textcolor{green}{10}: frequency domain $\mathbf{E}$-field  \\
\textcolor{green}{11}: frequency domain $\mathbf{H}$-field\\
\textcolor{green}{12}: frequency domain electrical current($\mathbf{I}$)\\
\textcolor{green}{13}: frequency domain electrical current density($\vec{\mathbf{J}}$)\\

\textcolor{green}{20}:frequency domain local SAR \\

If this parameter is not set, the default '0' will be adopted. 
\end{FontPara}


\begin{FontPara}{DumpMode}
\textcolor{green}{0}:no-interpolation \\
\textcolor{green}{1}:node-interpolation (default DumpMode ,if it is not defined)\\
\textcolor{green}{2}:cell-interpolation

User are advised to use no- or cell-interpolation for $\mathbf{E}$-field calculation to avoid false result;use no- or node-interpolation for $\mathbf{H}$-field calculation.

\end{FontPara}

\begin{FontPara}{FileType}
  0 : vtk-file dump  (by default)\\
  1 : hdf5-file dump
\end{FontPara}

\begin{FontPara}{Frequency}
Frequency of interest for dump. 
\end{FontPara}

\end{FontDescr}

\begin{FontDescr}{Optional Arguments:}
\begin{FontPara}{SubSampling}
Field domain sub-sampling. For example, '2,2,4'. It means ?? more explanation is needed  ???
\end{FontPara}

\begin{FontPara}{OptResolution}
Field domain dump resolution, For example,'10' or '10,20,5'. It means?? more explanation is needed  ???
\end{FontPara}

\end{FontDescr}

\begin{FontDescr}{Examples:}
\begin{lstlisting} 
AddDump(CSX,'Et');
CSX = AddBox(CSX,'Et',0,[0 0 0],[100 100 200]); 
\end{lstlisting}

AddDump is added into \hyperref[CSX]{\matv{CSX}} with name Et.A box primitive with name Et has been assigned for time domain E-field(0) storage.   

\begin{lstlisting} 
AddDump(CSX,'Ef',DumpType, 10, 'Frequency',[1e9 2e9]);
CSX = AddBox(CSX,'Ef',10,[0 0 0],[100 100 200])
\end{lstlisting}

Frequency domain E-field(10)with name Ef has been 'dumped' into \hyperref[CSX]{\matv{CSX}}. The frequency of interest are 100MHz and 200MHz. These properties have been assigned to the box primitive.      

\begin{lstlisting} 
CSX=AddDump(CSX,'SAR','DumpType',20,'FileType',0,...
    'DumpMode',2,'Frequency',300e6);
CSX=AddBox(CSX,'SAR',0,[0 0 0],[100 100 200]);
\end{lstlisting}

SAR(20) at frequency 300MHz has been 'dumped' into \hyperref[CSX]{\matv{CSX}} in vtk. file type.These properties have been assigned to the box primitive.          

\end{FontDescr}


