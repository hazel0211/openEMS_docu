\section{Properties}\label{csx_prop} 
This section describes the physical and non-physical properties of a model and has to be defined before the primitive properties. 


The physical properties which can be used: 
 
\begin{enumerate}
\item Material Properties(refer to \ref{subsection_material_prop})
\item Metal (refer to \ref{subsection_metal})
\item Discretized material definition (refer to \ref{subsection_disc_material})
\item Dispersive Material(refer to \ref{subsection_dispersive_material})
\item LumpedElement(refer to \ref{subsection_lumpedelement})
\end{enumerate}


The non-physical properties which can be used: 

\begin{enumerate}
\item Electrode (refer to \ref{subsection_Electrode})
\item ProbeBox(refer to \ref{subsection_ProbeBox})
\item DumpBox (refer to \ref{subsection_DumpBox})
\end{enumerate}


\subsection{General property setup}\label{subsection_gprop_setup}
To add a physical or non-physical property into \hyperref[CSX]{\matv{CSX}}, it has a common structure as shown below:

\begin{FontDescr}{Syntax:}
\begin{lstlisting} 
CSX=Add<..property..>(CSX,name,...)
\end{lstlisting} 
\end{FontDescr}

\begin{FontDescr}{Description:}
\matv{CSX} 
\phantomsection \label{CSX}
\begin{myindentpar}
\matv{CSX} describes geometrical objects and their physical or non-physical properties in .xml file.
\end{myindentpar}

\texttt{name} 
\begin{myindentpar}
\texttt{name} is given by user and must be matched with that is mentioned in the syntax corresponding to this defined property.  
\end{myindentpar}
\end{FontDescr}


\subsection{Material Properties}\label{subsection_material_prop}

\begin{FontNameFunct}{AddMaterial()}
\end{FontNameFunct}

\begin{FontDescr}{Purpose:}
 To add specified material into \hyperref[CSX]{\matv{CSX}}.
\end{FontDescr}

\begin{FontDescr}{Syntax:}
 \begin{lstlisting}
CSX = AddMaterial(CSX, name,varargin)
 \end{lstlisting}
\end{FontDescr}

\begin{FontDescr}{Description:}
  This function adds a material property to \hyperref[CSX]{\matv{CSX}} with the given name.
\end{FontDescr}
\begin{FontDescr}{Optional Arguments:}
\begin{myindentpar} 
  \matv{Isotropy} :If it is set to be '0', an anistropy object is defined. 
\end{myindentpar}  
\end{FontDescr}

\begin{FontNameFunct}{SetMaterialProperty()}
 \end{FontNameFunct}
 
\begin{FontDescr}{Purpose:}
The material properties such as relative conductivity, permittivity and permeability can be defined with this function.
\end{FontDescr}

 \begin{FontDescr}{Syntax:}
  \begin{lstlisting}
 CSX = SetMaterialProperty(CSX, name, varargin)
  \end{lstlisting}
 \end{FontDescr}
  
 \begin{FontDescr}{Description:} 
 \begin{FontDescr}{Optional Arguments:}
  It has to be defined by user to specific the material properties.\\ 

   \textcolor{varcol}{Epsilon($\varepsilon_{r}$)}\begin{myindentpar}
     Relative permittivity [1 by default].\end{myindentpar} 

 \textcolor{varcol}{Mue($\mu_{r}$)} \begin{myindentpar}
Relative magnetic permeability[1 by default].\end{myindentpar}   

  \textcolor{varcol}{Kappa($\kappa$)} \begin{myindentpar} Electric conductivity[$\frac{S}{m}$].
\end{myindentpar} 

  \textcolor{varcol}{Sigma($\sigma$)} \begin{myindentpar}Magnetic conductivity [$\frac{\Omega}{m}$].\end{myindentpar} 

  \begin{FontPara}{Density} Material mass density[$\frac{Kg}{m^{3}}$] \end{FontPara}

   They are some frequency dependent properties which are used in describing dispersive material.Refer to \ref{subsection_dispersive_material}. 
  
 \end{FontDescr}
 \end{FontDescr}


\begin{FontNameFunct}{SetMaterialWeight}
\end{FontNameFunct}

\begin{FontDescr}{Syntax:}
  \begin{lstlisting}
 CSX = SetMaterialWeight(CSX, name, varargin)
  \end{lstlisting}
\end{FontDescr}  

\begin{FontDescr}{Purpose:}
To weight material property with known function
\end{FontDescr}

\begin{FontDescr}{Description:}
It weights a material property with given weighting function by using the variables.
\end{FontDescr}

\begin{FontDescr}{Optional Arguments:}
 The following variables can be used to define the weighting function:

\begin{FontPara}{x}{y}{z}
the distance from x,y or z along its axis to the origin
\end{FontPara}

\begin{FontPara}{rho}
the distance z-axis. It equals to $\sqrt{x^{2}+y^{2}}$
\end{FontPara}

\begin{FontPara}{r}
the distance from point to origin. It equals to $\sqrt{x^{2}+y^{2}+z^{2}}$ 
\end{FontPara}

\begin{FontPara}{a}
the polar angle as in cylindrical and spherical coordinate systems. $a=\arctan(y,x)$
\end{FontPara}

\begin{FontPara}{t}
the azimuthal angle in spherical coordinate system . $t=\arccos (z,r)$
\end{FontPara}

\end{FontDescr}


\begin{FontDescr}{Examples:}

\begin{lstlisting}
 CSX = AddMaterial( CSX, 'RO3003' );
 CSX = SetMaterialProperty( CSX, 'RO3010', 'Epsilon', 3, 'Mue', 1 );
\end{lstlisting}
 
 The first syntax adds a dielectric material named Ro3003. It has relative permittivity of 3 and relative permeability of 1. If lossy dielectric is introduced, one can set \matv{Kappa}
($\kappa$)to its correponding value with known lost tangent ($\delta$): \begin{equation}
\kappa=2*pi*freq*\varepsilon*\tan(\delta)
\end{equation} 
\begin{lstlisting}
 CSX = AddMaterial( CSX, 'RO3003' );
 CSX = SetMaterialProperty( CSX, 'RO3010', 'Epsilon', 3,'Kappa',Kappa);
\end{lstlisting}

The following example shows anisotropic material property: 

\begin{lstlisting} 
  CSX = AddMaterial( CSX, 'sheet','Isotropy',0);
  CSX = SetMaterialProperty(CSX, 'sheet', 'Kappa', [0 0 kappa]);
\end{lstlisting}

The anisotropic material named sheet has zero \matv{Kappa} at x- and y- direction but values 'kappa' at z-direction, where 'kappa' is predefined. If cylindrical coordinate system has been used, then the material sheet has only value at z-direction but not at radial and azimuthal direction. 
    
\begin{lstlisting} 
CSX = AddMaterial(CSX, 'abc');
CSX = SetMaterialProperty(CSX, 'abc', 'Epsilon', 1);
CSX = SetMaterialWeight(CSX, 'abc', 'Epsilon', ['(sin(4*z / 1000 *2*pi)>0)+1']);
\end{lstlisting}

A material named 'abc' has been added into \hyperref[CSX]{\matv{CSX}}.Its relative permittivity has been weighted with function sin depends on its z-position. 

\end{FontDescr}


\subsection{Metal}\label{subsection_metal}
 This section shows how a metal/PEC material is introduced into
 \hyperref[CSX]{\matv{CSX}}.
 
\begin{FontNameFunct}{AddMetal}
\end{FontNameFunct}

\begin{FontDescr}{Syntax:}
  \begin{lstlisting}
 CSX = AddMetal(CSX, name)
  \end{lstlisting}
\end{FontDescr}  

\begin{FontDescr}{Description:}
This function introduces perfect electric conductor of no loss into \hyperref[CSX]{\matv{CSX}}.
\end{FontDescr}

 
\begin{FontNameFunct}{AddConductingSheet}
\end{FontNameFunct}

\begin{FontDescr}{Purpose:}
To add a lossy conducting material into \hyperref[CSX]{\matv{CSX}}.
\end{FontDescr}

\begin{FontDescr}{Syntax:}
  \begin{lstlisting}
CSX = AddConductingSheet(CSX, name, conductivity, thickness)
  \end{lstlisting}
\end{FontDescr} 

\begin{FontDescr}{Description:}

\begin{FontPara}{conductivity}
It is given by user according to the conductivity of metal being used. The most frequent used metal is copper and its conductivity is 58e6$Sm^{-1}$.    
\end{FontPara}

 \begin{FontPara}{thickness}
 It defines the thickness of lossy metal sheet.     
 \end{FontPara}
\end{FontDescr} 

\begin{FontDescr}{Examples:} 

\begin{lstlisting} 
CSX = AddMetal(CSX,'metal'); 
\end{lstlisting}
This syntax adds PEC material into \hyperref[CSX]{\matv{CSX}} with the name 'metal'. \\

\begin{lstlisting} 
CSX = AddConductingSheet(CSX,'copper',56e6,70e-6);
\end{lstlisting}
This syntax adds a conducting sheet of 70$\mu$m into \hyperref[CSX]{\matv{CSX}}. The assigned metal is named copper and has conductivity of 58e6$Sm^{-1}$.  

 \end{FontDescr} 
 
 
 
 
 
\subsection{Discretized material definition}\label{subsection_disc_material}
 \input{chapter/SEC_CSXCAD_Setup/properties_inhomo_phantom}

\subsection{Dispersive Material}\label{subsection_dispersive_material}
 %Dispersive Material
User can model a Drude type dispersive material by function \texttt{AddLorentzMaterial()}. Dispersive material has frequency dependent relative permittivity($\varepsilon_{r}$) and permeability($\mu_{r}$), and they can be defined with function \texttt{ SetMaterialProperty()}.
\begin{equation}
\varepsilon_{r}(\omega)=\varepsilon_{r}*(1-\dfrac{\omega_{\varepsilon}^2}{\omega*(\omega-j/t_{\varepsilon})})
\end{equation}
\begin{equation}
\mu_{r}(\omega)=\mu_{r}*(1-\dfrac{\omega_{\mu}^2}{\omega*(\omega-j/t_{\mu})})
\end{equation}
where \begin{myindentpar}
\begin{itemize}
\item $\omega_{\varepsilon}$: the respective electric angular plasma frequency
\item $t_{\varepsilon}$: the electric relaxation time 
\item $\omega_{\mu}$:the respective magnetic angular plasma frequency
\item $t_{\mu}$:the magnetic relaxation time
\end{itemize}
\end{myindentpar}

\begin{FontNameFunct}{AddLorentzMaterial}
\end{FontNameFunct}

\begin{FontDescr}{Syntax:}
  \begin{lstlisting}
 CSX = AddLorentzMaterial(CSX, name)
  \end{lstlisting}
\end{FontDescr}

\begin{FontDescr}{Description:}
  This syntax adds a Drude type dispersive material model with the given name.\\ 
\end{FontDescr} 
  

\begin{FontNameFunct}{SetMaterialProperty}  
\end{FontNameFunct}\\
In this section the frequency dependent properties $\varepsilon_{r}(\omega)$ and $\mu_{r}(\omega)$ will be defined. Refer to \ref{subsection_material_prop} for non-dispersive material properties setup. 

 \begin{FontDescr}{Syntax:}
  \begin{lstlisting}
 CSX = SetMaterialProperty(CSX, name, varargin)
  \end{lstlisting}
 \end{FontDescr}
 
 \begin{FontDescr}{Arguments:}  
 
  \begin{FontPara}{EpsilonPlasmaFrequency}
  Electric plasma frequency($f_{\varepsilon}$). It equals to   $\omega_{\varepsilon}/2\pi$. 
  \end{FontPara} 
  \begin{FontPara}{MuePlasmaFrequency}($f_{\mu}$)
  Magnetic plasma frequency.It equals to $\omega_{\mu}/2\pi$.   
  \end{FontPara}
  \begin{FontPara}{EpsilonRelaxTime}   
   Electric plasma relaxation time($t_{\varepsilon}$). Smaller number results in greater losses or alternatively set it to '0' for lossless case. 
   \end{FontPara}
  \begin{FontPara}{MueRelaxTime}  
  Magnetic plasma relaxation time($t_{\mu}$).Smaller number results in greater losses or alternatively set it to '0' for lossless case.
 \end{FontPara}

For higher order Drude type, the above mentioned material constant is modified to: 

 \textcolor{varcol}{\texttt{EpsilonPlasmaFrequency$\_<n>$}}
\begin{myindentpar}
 n-th electric plasma frequency.
\end{myindentpar} 
  \textcolor{varcol}{\texttt{MuePlasmaFrequency$\_<n>$} }  
  \begin{myindentpar}
  n-th magnetic plasma frequency. 
  \end{myindentpar} 
  \textcolor{varcol}{\texttt{EpsilonRelaxTime$\_<n>$}} 
   \begin{myindentpar}
   n-th electric plasma relaxation time($t_{\varepsilon}$).   \end{myindentpar} 
 \textcolor{varcol}{\texttt{MueRelaxTime$\_<n>$}} 
\begin{myindentpar}
n-th magnetic plasma relaxation time($t_{\mu}$).
 \end{myindentpar}  
\end{FontDescr}  


\begin{FontDescr}{Example:}  
\begin{lstlisting} 
 CSX = AddLorentzMaterial(CSX,'drude');
 CSX = SetMaterialProperty(CSX,'drude','Epsilon',5,
 'EpsilonPlasmaFrequency',5e9,'EpsilonRelaxTime',1e-9);
 CSX = SetMaterialProperty(CSX,'drude','Mue',5,
 'MuePlasmaFrequency',5e9,'MueRelaxTime',1e-9);
\end{lstlisting}
  
  A Lorent material of name 'drude' has been defined. Its $\varepsilon_{r}$ and $\mu_{r}$ are set to 5 ,$f_{\varepsilon}$ and $f_{\mu}$ are both at 500M Hz while $t_{\varepsilon}$ and $t_{\mu}$ value 1e-9. 
 \end{FontDescr}
  
  
  

\subsection{LumpedElement}\label{subsection_lumpedelement}
Lumped elements like capacitor , resistor and inductor are important for tuning or matching purpose. These elements can be added into CSX with specified value and direction with function \texttt{AddLumpedElement()}. 

\begin{FontNameFunct}{AddLumpedElement}
\end{FontNameFunct}


\begin{FontDescr}{Syntax:}
  \begin{lstlisting}
CSX = AddLumpedElement(CSX, name, direction, varargin)
  \end{lstlisting}
 \end{FontDescr} 
 
\begin{FontDescr}{Description:}
\begin{FontPara}{direction}
This sets orientation of lumped elements. If a lumped element is aimed to connect two faces which are in z-direction separated, then z-orientation lumped element has to be chosen. 
 \begin{itemize}
 \item \textcolor{green}{0} :x-orientation 
 \item \textcolor{green}{1} :y-orientation
 \item \textcolor{green}{2} :z-orientation
 \end{itemize}
 \end{FontPara}
\end{FontDescr}

  
\begin{FontDescr}{Arguments:} 
 To define which lumped element to be added. 
 
 \begin{FontPara}{R}
 Lumped resistor
 \end{FontPara}
 \begin{FontPara}{C}
 Lumped capacitor
 \end{FontPara}
 \begin{FontPara}{L}
 Lumped inductor
 \end{FontPara}
 \begin{FontPara}{Caps}
It can be set to 0 or 1 to (de)activate lumped capacitor.(1=default) 
 \end{FontPara}               
 
\end{FontDescr}
 
 
\begin{FontDescr}{Example:}

\begin{lstlisting} 
CSX = AddLumpedElement( CSX, 'Capacitor', 0, 'Caps', 1, 'C', 5e-12 );
\end{lstlisting}
 A lumped capacitor in x-direction with 5pF has been added and activated. It is named as 'Capacitor'. 
\end{FontDescr}
 
non-Physical CSXProperties
\subsection{Electrode}\label{subsection_Electrode}  
Previously, the time or frequency signal excitation has been introduced in chapter~\ref{sec:FDTD_Excitation}. In this section, its geometrical distribution will be discussed. An excitation signal must be set (into \matv{FDTD}) before specifying its geometrical structure. 

Either a voltage source($U$) or a current source ($I$) can be defined with field property in function \texttt{AddExcitation()}. A voltage source can be seen as 
$\mathbf{E}$-field excitation while current source as $\mathbf{H}$-field excitation. 

    \begin{equation}
    U=\oint_{C}\vec{\mathbf{E}}.d\overrightarrow{s}
    \end{equation}
    \begin{equation}
    I=\oint_{C}\vec{\mathbf{H}}.d\overrightarrow{s}
    \end{equation}\\

\begin{FontNameFunct}{AddExcitation()}
\end{FontNameFunct}

\begin{FontDescr}{Purpose:}
Create an $\mathbf{E}$-field or $\mathbf{H}$-field excitation.
\end{FontDescr}

\begin{FontDescr}{Syntax:}
  \begin{lstlisting}
  CSX = AddExcitation(CSX, name, type, excite, varargin)
  \end{lstlisting}
\end{FontDescr}

\begin{FontDescr}{Description:}
 
A soft type field sums the from time to time updated field of the structure and field of the source while a hard type field calculate only the field of the source. It can be understood as a field locates within air enviroment for soft type field or lies near to a metal(PEC/PMC) condition for a hard type field(the total field is same as the reflected field of the source field).  

$\mathbf{E}$-field will be calculated as: 

Soft-type:
\begin{equation}
\vec{\mathbf{E}}=\vec{\mathbf{E}}_{\text{fdtd}}+\vec{\mathbf{E}}_{\text{src}} 
\end{equation}   

Hard-type:
\begin{equation}
\vec{\mathbf{E}}=\vec{\mathbf{E}}_{\text{src}}     
\end{equation}   

The same applies to $\mathbf{H}$-field.

\begin{FontPara}{type}
\textcolor{green}{0} : $\mathbf{E}$-field soft excitation \\
\textcolor{green}{1} : $\mathbf{E}$-field hard excitation\\
\textcolor{green}{2} : $\mathbf{H}$-field soft excitation\\
\textcolor{green}{3} : $\mathbf{H}$-field hard excitation\\
\textcolor{green}{10}: plane wave excitation
\end{FontPara}

\begin{FontPara}{excite}\phantomsection\label{excite_dir}
Row vector for direction of excitation.
\begin{myindentpar}
  x-or r-direction: [1 0 0] \\
  y-direction: [0 1 0] \\
  z-direction: [0 0 1] 
\end{myindentpar}
\end{FontPara}

\end{FontDescr}

\begin{FontDescr}{Additional Arguments:}

\begin{FontPara}{'Delay'} \phantomsection\label{delay}
Setup an excitation time delay in seconds. If a $x$ phase delay(in degree) at frequency $f_{1}$ is desired, 
\begin{equation}
delay= \dfrac{x}{360*f_{1}}
\end{equation}   
\end{FontPara}

\begin{FontPara}{'PropDir'} 
Only for plane wave type excitation use. Describe the direction of plane wave propagation.
\end{FontPara}
\end{FontDescr}

\begin{FontDescr}{Example:}
\begin{lstlisting} 
 CSX = AddExcitation( CSX, 'Dipole', 1, [1 0 0] );
\end{lstlisting}
An E-Field hard excitation in x-direction has been defined in \hyperref[CSX]{\matv{CSX}} with the name Dipole. 
\end{FontDescr}


\begin{FontNameFunct}{AddPlaneWaveExcite()}
\end{FontNameFunct}

\begin{FontDescr}{Purpose:}
To create a plane wave excitation in air/vacuum and
completely surround a structure. The plane wave excitation must not intersect with any kind of material. Only box property(refer to ~) can be used to describe this excitation. 

\end{FontDescr}

\begin{FontDescr}{Syntax:}
\begin{lstlisting}
CSX = AddPlaneWaveExcite(CSX, name, k_dir, E_dir, <f0, varargin>)
\end{lstlisting}
\end{FontDescr}

\begin{FontDescr}{Description:}

\textcolor{varcol}{k$\_$dir}
\begin{myindentpar}%[3em]
A row vector to describe direction of wave propagation($\vec{\mathbf{S}}$).
It must be orthorgonal to E$\_$dir. Refer to \hyperref[excite_dir]{\matv{'excite'}} in previous function for vector description.
\end{myindentpar}

\begin{equation}
\vec{\mathbf{S}}=\vec{\mathbf{E}}\times\vec{\mathbf{H}}
\end{equation}

\textcolor{varcol}{E$\_$dir} \phantomsection\label{Edir}
\begin{myindentpar}%[3em]
A row vector to describe direction of polarisation. It equals to direction of $\mathbf{E}$-field($\vec{\mathbf{E}}$). Refer to \hyperref[excite_dir]{\matv{'excite'}} in previous function for vector description. 
\end{myindentpar}
\end{FontDescr}

\begin{FontDescr}{Additional Argument:}
\begin{FontPara}{f0}
Frequency for numerical phase velocity compensation
\end{FontPara}
\end{FontDescr}

\begin{FontDescr}{Example:}
\begin{lstlisting} 
 k_dir = [1 0 0]; 
 E_dir = [0 0 1]; 
 f0 = 500e6;      
 CSX = AddPlaneWaveExcite(CSX, 'plane_wave', k_dir, E_dir, f0);
\end{lstlisting}
A plavewave excitation has been added into\hyperref[CSX]{\matv{CSX}} with the name plane$\_$wave. It is z-polarised and propagate in x-direction.  
\end{FontDescr}


\begin{FontNameFunct}{AddLumpedPort()}
\end{FontNameFunct}

\begin{FontDescr}{Purpose:}
Add a 3D geometry as excitation element.
\end{FontDescr}

\begin{FontDescr}{Syntax:}
  \begin{lstlisting}
 [CSX] = AddLumpedPort( CSX, prio, portnr, R, start, stop, dir, excitename, varargin )
  \end{lstlisting}
\end{FontDescr}

\begin{FontDescr}{Description:}
 A lumped port has been added into \hyperref[CSX]{\matv{CSX}} with the give name. 
 
\begin{FontPara}{prio}
Priority for substrate and probe boxes of the lumped port. The element of higher priority will be simulated whenever two elements of different priority are overlapping.  
\end{FontPara}

\begin{FontPara}{portnr}
Integer to represent port number. If more than one port is defined, different port numbers are recommended. 
\end{FontPara}

\begin{FontPara}{R}
Port internal resistance. For matching with the transmission line, it should be set to 50$\Omega$. 
\end{FontPara}

\begin{FontPara}{start}
A row vector to represent the starting point of the 3D geometry.
\end{FontPara}

\begin{FontPara}{stop}
A row vector to represent the end point of the 3D geometry.
\end{FontPara}

\begin{FontPara}{dir}
Direction of port. Refer to \hyperref[excite_dir]{\matv{'excite'}} in previous function.
\end{FontPara}

\begin{FontPara}{excitename}
If it is defined, the port is switched on. Else, it is only a(lossy)conducting path.
\end{FontPara}

\end{FontDescr} 

\begin{FontDescr}{Additional Argument:}
\begin{FontPara}{'Delay'}
Setup an excitation time delay in seconds. Refer to \hyperref[delay]{\matv{'delay'}} in previous function.
\end{FontPara}
\end{FontDescr} 

\begin{FontDescr}{Example:}

\begin{lstlisting} 
 start = [-0.1 -0.1 0];
 stop  = [ 0.1 0.1 5];
 [CSX] = AddLumpedPort(CSX, 5 ,1 , 50, start, stop, [0 0 1], 'excite');
\end{lstlisting}
A z-direction lumped port of dimension 0.2x0.2x5 has been switched on. It has priority of 5 and internal resistance of 50$\Omega$. If the 'excite' is omitted, this port is switched off. 
   
\begin{lstlisting} 
 start = [-0.1 -0.1 0];
 stop  = [ 0.1 0.1 5];
 [CSX] = AddLumpedPort(CSX, 5 ,1 , 50, start, stop, [0 0 1], 'excite','Delay',1/4/300e6 );
\end{lstlisting}
The port with $90^{0}$ phase shift has been switched on. 
\end{FontDescr}


\begin{FontNameFunct}{AddMSLPort()}
\end{FontNameFunct}

\begin{FontDescr}{Purpose:}
Create a MSL port with defined material property. 
\end{FontDescr}

\begin{FontDescr}{Syntax:}
  \begin{lstlisting}
[CSX,port] = AddMSLPort( CSX, prio, portnr, materialname, start, stop, dir, evec, varargin )
  \end{lstlisting}
\end{FontDescr}

\begin{FontDescr}{Description:}
\begin{FontPara}{materialname}
Property for the MSL. Function \texttt{AddMetal()}can be used to defined its property. 
\end{FontPara}

\begin{FontPara}{dir}
A number to represent wave propagation direction.\\
\textcolor{green}{0}= x-direction\\
\textcolor{green}{1}= y-direction\\
\textcolor{green}{2}= z-direction\\
\end{FontPara}

\begin{FontPara}{evec}
Refer to \hyperref[Edir]{\matv{Edir}} of function \texttt{AddLumpedPort()}. 
\end{FontPara}
\end{FontDescr}

\begin{FontDescr}{Additional Arguments:}
\begin{FontPara}{'ExcitePort'}
An excitation name.If it is defined, the port is switched on.
\end{FontPara}
\begin{FontPara}{'FeedShift'}
Shift from start to port by a given distance in drawing units. Default is 0. Only active if 'ExcitePort' is set.
\end{FontPara}
\begin{FontPara}{'MeasPlaneShift'}
Shift the measure plane to a distance specified in drawing unit. Default is the middle of start and stop vector. Only active if 'ExcitePort' is set.
\end{FontPara}

\textcolor{varcol}{'Feed$\_$R'}
Port resistance. Default is zero. It only active if 'ExcitePort' is set.

\end{FontDescr}

\begin{FontDescr}{Example:}
\begin{lstlisting} 
start=[0 0 height]; 
stop = [length width 0]; 
CSX = AddMetal( CSX, 'metal' ); %create a PEC called 'metal'
[CSX,port] = AddMSLPort( CSX, 0, 1, 'metal', start,
             stop,0, [0 0 -1],'ExcitePort','excite',     
              'Feed_R', 50 )
\end{lstlisting} 

This defines a MSL in x-direction (dir=0) with an $\mathbf{E}$-field excitation in -z-direction. The excitation is placed at x=start(1) and 
the wave travels toward x=stop(1). The MSL-metal is created in xy-plane at z=0 height by function \hyperref[addmetal]{\matv{'AddMetal()'}}.
    
\end{FontDescr}
\subsection{ProbeBox}\label{subsection_ProbeBox}  
\input{chapter/SEC_CSXCAD_Setup/probebox}
\subsection{DumpBox}\label{subsection_DumpBox}
This section introduces 'Dumpbox' property to store $\mathbf{E}$,$\mathbf{H}$-field, electrical current , current density and SAR(Specific Absorption Rate) in a specified box.  


\begin{FontNameFunct}{AddDump()}
\end{FontNameFunct}

\begin{FontDescr}{Purpose:}
Add Dumpbox property into 
\matv{CSX}\phantomsection\label{CSX} with the given name. 
\end{FontDescr}

\begin{FontDescr}{Syntax:}
\begin{lstlisting} 
 CSX = AddDump(CSX, name, varargin)
\end{lstlisting}
\end{FontDescr}

\begin{FontDescr}{Description:}

\begin{FontPara}{DumpType}
\textcolor{green}{0}: time domain $\mathbf{E}$-field \\
\textcolor{green}{1}: time domain $\mathbf{H}$-field \\
\textcolor{green}{2}: time domain electrical current($\mathbf{I}$)  \\
\textcolor{green}{3}: time domain electrical current density($\vec{\mathbf{J}}$)\\

\textcolor{green}{10}: frequency domain $\mathbf{E}$-field  \\
\textcolor{green}{11}: frequency domain $\mathbf{H}$-field\\
\textcolor{green}{12}: frequency domain electrical current($\mathbf{I}$)\\
\textcolor{green}{13}: frequency domain electrical current density($\vec{\mathbf{J}}$)\\

\textcolor{green}{20}:frequency domain local SAR \\

If this parameter is not set, the default '0' will be adopted. 
\end{FontPara}


\begin{FontPara}{DumpMode}
\textcolor{green}{0}:no-interpolation \\
\textcolor{green}{1}:node-interpolation (default DumpMode ,if it is not defined)\\
\textcolor{green}{2}:cell-interpolation

User are advised to use no- or cell-interpolation for $\mathbf{E}$-field calculation to avoid false result;use no- or node-interpolation for $\mathbf{H}$-field calculation.

\end{FontPara}

\begin{FontPara}{FileType}
  0 : vtk-file dump  (by default)\\
  1 : hdf5-file dump
\end{FontPara}

\begin{FontPara}{Frequency}
Frequency of interest for dump. 
\end{FontPara}

\end{FontDescr}

\begin{FontDescr}{Optional Arguments:}
\begin{FontPara}{SubSampling}
Field domain sub-sampling. For example, '2,2,4'. It means ?? more explanation is needed  ???
\end{FontPara}

\begin{FontPara}{OptResolution}
Field domain dump resolution, For example,'10' or '10,20,5'. It means?? more explanation is needed  ???
\end{FontPara}

\end{FontDescr}

\begin{FontDescr}{Examples:}
\begin{lstlisting} 
AddDump(CSX,'Et');
CSX = AddBox(CSX,'Et',0,[0 0 0],[100 100 200]); 
\end{lstlisting}

AddDump is added into \hyperref[CSX]{\matv{CSX}} with name Et.A box primitive with name Et has been assigned for time domain E-field(0) storage.   

\begin{lstlisting} 
AddDump(CSX,'Ef',DumpType, 10, 'Frequency',[1e9 2e9]);
CSX = AddBox(CSX,'Ef',10,[0 0 0],[100 100 200])
\end{lstlisting}

Frequency domain E-field(10)with name Ef has been 'dumped' into \hyperref[CSX]{\matv{CSX}}. The frequency of interest are 100MHz and 200MHz. These properties have been assigned to the box primitive.      

\begin{lstlisting} 
CSX=AddDump(CSX,'SAR','DumpType',20,'FileType',0,...
    'DumpMode',2,'Frequency',300e6);
CSX=AddBox(CSX,'SAR',0,[0 0 0],[100 100 200]);
\end{lstlisting}

SAR(20) at frequency 300MHz has been 'dumped' into \hyperref[CSX]{\matv{CSX}} in vtk. file type.These properties have been assigned to the box primitive.          

\end{FontDescr}


