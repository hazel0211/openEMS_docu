Introduce a spherical shell into\hyperref[CSX]{\matv{CSX}}. 

\begin{FontNameFunct}{AddSphericalShell()}
\end{FontNameFunct}

\begin{FontDescr}{Purpose:}
To add a spherical shell in\hyperref[CSX]{\matv{CSX}} by defining its radius, center point and shell width. 
\end{FontDescr}

\begin{FontDescr}{Syntax:}
\begin{lstlisting}
CSX = AddSphericalShell(CSX, propName, prio, center, rad, shell_width, varargin) 
\end{lstlisting}
\end{FontDescr}

\begin{FontDescr}{Description:}
The parameters are defined same as those in \texttt{AddSphere}(subsection~\ref{sphere}) except the following: \\
\textcolor{varcol}{shell$\_$width}
\begin{myindentpar} Width of the shell. The inner radius and outer radius of shell is calculated same as \hyperref[rin]{$r_{in}$} and \hyperref[rout]{$r_{out}$} of section \ref{cylindershell}.
\end{myindentpar} 
\end{FontDescr}

\begin{FontDescr}{Optional Arguments:}
The standard trasformation (rotation,translation,scaling) mentioned in  \hyperref[prim_transform]{\matv{'Transform'}} of \texttt{AddBox}.   
\end{FontDescr}

\begin{FontDescr}{Examples:}

\begin{lstlisting} 
 CSX = AddMetal(CSX,'metal'); 
 CSX = AddSphericalShell(CSX,'metal',10,[0 0 0],50,10);
\end{lstlisting}
A metallic spherical shell of radius 50 has been added into\hyperref[CSX]{\matv{CSX}}. The thickness of shell is 10. So, the inner radius of shell is 50-10/2=45 while outer radius of shell is 50+10/2=55.   

\begin{lstlisting} 
 CSX = AddMetal(CSX,'metal'); 
 CSX = AddSphericalShell(CSX,'metal',10,[0 0 0],50,10,'Transform',{'Scale','3,3,3'});
\end{lstlisting}
The above mentioned spherical shell has been scaled with factor 3. It is 3 times larger than original size. 

\end{FontDescr}