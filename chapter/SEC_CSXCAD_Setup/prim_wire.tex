Introduce a curve with defined radius into \hyperref[CSX]{\matv{CSX}} with matlab function \texttt{AddWire} and assign a property to it. 

\begin{FontNameFunct}{AddWire()}
\end{FontNameFunct}


\begin{FontDescr}{Purpose:}
Add wire to \hyperref[CSX]{\matv{CSX}} by defining its coordinate arrays and radius. Take note that wire is not a solid cylinder!  
\end{FontDescr}

\begin{FontDescr}{Syntax:}
\begin{lstlisting} 
 CSX = AddWire(CSX, propName, prio, points, wire_rad, varargin)
\end{lstlisting}
\end{FontDescr}

\begin{FontDescr}{Description:}

\begin{FontPara}{propName}
Refer to \hyperref[prim_Name]{propName} in \texttt{AddBox}. 
\end{FontPara}

\begin{FontPara}{points}
Refer to parameter \hyperref[points_curve]{\matv{points}} in section \ref{curve}.
\end{FontPara}

\textcolor{varcol}{wire$\_$rad}
\begin{myindentpar}
Wire radius. 
\end{myindentpar}
\end{FontDescr}

\begin{FontDescr}{Optional Arguments:}
The standard trasformation (rotation,translation,scaling) mentioned in  \hyperref[prim_transform]{\matv{'Transform'}} of \texttt{AddBox}.   
\end{FontDescr}

\begin{FontDescr}{Examples:}
\begin{lstlisting} 
%first point
     points(1,1) = 0;points(2,1) = 5;points(3,1) = 0; 
%second point
     points(1,2) = 0;points(2,2) = 5;points(3,2) = 100; 
     CSX = AddMetal(CSX,'metal'); 
     CSX = AddWire(CSX,'metal',10, points,2);
\end{lstlisting}
This example creates a metallic 100 unit long wire.This wire is hollow cylinder with defined radius and very thin shell width.   
\end{FontDescr}
