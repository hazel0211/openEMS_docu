Lumped elements like capacitor , resistor and inductor are important for tuning or matching purpose. These elements can be added into CSX with specified value and direction with function \texttt{AddLumpedElement()}. 

\begin{FontNameFunct}{AddLumpedElement}
\end{FontNameFunct}


\begin{FontDescr}{Syntax:}
  \begin{lstlisting}
CSX = AddLumpedElement(CSX, name, direction, varargin)
  \end{lstlisting}
 \end{FontDescr} 
 
\begin{FontDescr}{Description:}
\begin{FontPara}{direction}
This sets orientation of lumped elements. If a lumped element is aimed to connect two faces which are in z-direction separated, then z-orientation lumped element has to be chosen. 
 \begin{itemize}
 \item \textcolor{green}{0} :x-orientation 
 \item \textcolor{green}{1} :y-orientation
 \item \textcolor{green}{2} :z-orientation
 \end{itemize}
 \end{FontPara}
\end{FontDescr}

  
\begin{FontDescr}{Arguments:} 
 To define which lumped element to be added. 
 
 \begin{FontPara}{R}
 Lumped resistor
 \end{FontPara}
 \begin{FontPara}{C}
 Lumped capacitor
 \end{FontPara}
 \begin{FontPara}{L}
 Lumped inductor
 \end{FontPara}
 \begin{FontPara}{Caps}
It can be set to 0 or 1 to (de)activate lumped capacitor.(1=default) 
 \end{FontPara}               
 
\end{FontDescr}
 
 
\begin{FontDescr}{Example:}

\begin{lstlisting} 
CSX = AddLumpedElement( CSX, 'Capacitor', 0, 'Caps', 1, 'C', 5e-12 );
\end{lstlisting}
 A lumped capacitor in x-direction with 5pF has been added and activated. It is named as 'Capacitor'. 
\end{FontDescr}